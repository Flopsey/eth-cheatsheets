% Basic stuff
\documentclass[a4paper,10pt]{article}
\usepackage[nswissgerman]{babel}

% 3 column landscape layout with fewer margins
\usepackage[landscape, left=0.75cm, top=1cm, right=0.75cm, bottom=1.5cm, footskip=15pt]{geometry}
\usepackage{flowfram}
\ffvadjustfalse
\setlength{\columnsep}{1cm}
\Ncolumn{3}

\usepackage{dirtytalk}

% define nice looking boxes
\usepackage[many]{tcolorbox}

% a base set, that is then customised
\tcbset {
  base/.style={
    boxrule=0mm,
    leftrule=1mm,
    left=1.75mm,
    arc=0mm, 
    fonttitle=\bfseries, 
    colbacktitle=black!10!white, 
    coltitle=black, 
    toptitle=0.75mm, 
    bottomtitle=0.25mm,
    title={#1},
    breakable,
  }
}

\definecolor{theoretische-informatik}{rgb}{1.00, 0.55, 0.52}
\newtcolorbox{mainbox}[1]{
  colframe=theoretische-informatik, 
  base={#1}
}

\newtcolorbox{subbox}[1]{
  colframe=black!20!white,
  base={#1}
}

% Mathematical typesetting & symbols
\usepackage{amsthm, mathtools, amssymb} 
\usepackage{marvosym, wasysym}
\usepackage{derivative}
\usepackage{turnstile}
\allowdisplaybreaks
\DeclareMathOperator{\Bin}{Bin}

% Tables
\usepackage{tabularx, multirow}
\usepackage{booktabs}

% Make enumerations more compact
\usepackage{enumitem}
\setitemize{itemsep=0.5pt}
\setenumerate{itemsep=0.75pt}

% To include sketches & PDFs
\usepackage{graphicx}

% For hyperlinks
\usepackage{hyperref}
\hypersetup{
  colorlinks=true
}

% Metadata
\title{Zusammenfassung Theoretische Informatik}
\author{Florian Affolter}
\date{Herbstsemester 2022}

\begin{document}

\maketitle

\section{Alphabete, Wörter, Sprachen und die Darstellung von Problemen}
% Kapitel 2
\begin{mainbox}{Entscheidungsproblem}
Das \emph{Entscheidungsproblem $(\Sigma, L)$} für ein gegebenes Alphabet $\Sigma$ und eine gegebene Sprache $L \subseteq \Sigma^*$ ist, für jedes $x \in \Sigma^*$ zu entscheiden, ob
\[x \in L\ \text{oder}\ x \notin L \text{.}\]
Ein \emph{Algorithmus $A$} löst das Entscheidungsproblem $(\Sigma, L)$, falls für alle $x \in \Sigma^*$ gilt:
\[A(x) =
\begin{cases}
1, &\text{falls}\ x \in L,\\
0, &\text{falls}\ x \notin L \text{.}
\end{cases}
\]
Wir sagen auch, dass $A$ die Sprache $L$ \emph{erkennt}.
\end{mainbox}
Es existiert ein Algorithmus, der $L$ erkennt $\implies$ $L$ ist rekursiv.
\begin{subbox}{Optimierungsproblem}
Ein \emph{Optimierungsproblem} ist ein 6-Tupel $\mathcal{U} = (\Sigma_I, \Sigma_O, L, \mathcal{M}, \operatorname{cost}, \operatorname{goal})$, wobei:
\begin{itemize}
    \item $\Sigma_I$ ist ein Alphabet (genannt \emph{Eigabealphabet}),
    \item $\Sigma_O$ ist ein Alphabet (genannt \emph{Ausgabealphabet},
    \item $L \subseteq \Sigma_I^*$ ist die Sprache der \emph{zulässigen Eingaben},
    \item $\mathcal{M}$ ist eine Funktion von $L$ nach $\mathcal{P}(\Sigma_O^*)$, und für jedes $x \in L$ ist $\mathcal{M}(x)$ die \emph{Menge der zulässigen Lösungen für $x$},
    \item $\operatorname{cost}$ ist eine Funktion, $\operatorname{cost}:\: \bigcup_{x \in L} (\mathcal{M}(x) \times \{x\}) \to \mathbb{R}^+$, genannt \emph{Kostenfunktion},
    \item$\operatorname{goal} \in \{\text{Minimum}, \text{Maximum}\}$ ist das \emph{Optimierungsziel}.
\end{itemize}
\end{subbox}
\begin{subbox}{Aufzählungsalgorithmus}
    Sei $\Sigma$ ein Alphabet, und sei $L \subseteq \Sigma^*$. $A$ ist ein \emph{Aufzählungsalgorithmus für $L$}, falls $A$ für jede Eingabe $n \in \mathbb{N} - \{0\}$ die Wortfolge $x_1, x_2, \dots, x_n$ ausgibt, wobei $x_1, x_2, \dots, x_n$ die kanonisch $n$ ersten Wörter in $L$ sind.
\end{subbox}
\begin{mainbox}{Kolmogorov-Komplexität}
    Für jedes Wort $x \in (\Sigma_\textrm{bool})^*$ ist die \emph{Kolmogorov-Komplexität $K(x)$ des Wortes $x$} das Minimum der binären Längen der Pascal-Programme, die $x$ generieren.
\end{mainbox}
\begin{mainbox}{}
    % TODO: Fix vertical spacing
    \[K(x) \leq \left|x\right| + d,\quad x \in (\Sigma_\textrm{bool})^*,\, d\ \text{konstant}\]
\end{mainbox}
Ein Wort $x \in (\Sigma_\textrm{bool})^*$ heisst \emph{zufällig}, falls $K(x) \geq \left|x\right|$. Eine Zahl $n$ heisst \emph{zufällig}, falls $K(n) = K(\Bin(n)) \geq \left\lceil \log_2 (n + 1) \right\rceil - 1$.
\begin{subbox}{Primzahlsatz}
    \[\lim_{n \to \infty} \frac{\operatorname{Prim}(n)}{\frac{n}{\ln n}} = 1 \text{.}\]
\end{subbox}

\section{Endliche Automaten}
% Kapitel 3
\begin{mainbox}{Endlicher Automat}
    Ein \emph{(deterministischer) endlicher Automat (EA)} ist ein Quintupel $M = (Q, \Sigma, \delta, q_0, F)$, wobei
    \begin{itemize}
        \item $Q$ eine Menge von \emph{Zuständen} ist,
        \item $\Sigma$ ein Alphabet, genannt \emph{Eingabealphabet}, ist,
        \item $q_0 \in Q$ der Anfangszustand ist,
        \item $F \subseteq Q$ die \emph{Menge der akzeptierenden Zustände} ist und
        \item $\delta$ eine Funktion von $Q \times \Sigma$ nach $Q$ ist, die \emph{Übergangsfunktion} genannt wird.
    \end{itemize}
\end{mainbox}
\begin{itemize}
    \item \emph{Konfiguration}: Element aus $Q \times \Sigma^*$
    \begin{itemize}
        \item $(q_0, x)$: Startkonfiguration von $M$ auf $x$
        \item $(q, \lambda)$: Endkonfiguration
    \end{itemize}
    \item Schritt von $M$:
    \[(q, w) \sststile{M}{} (p, x) \iff w = ax,\quad a \in \Sigma\ \text{und}\ \delta(q, a) = p\]
    \item Die von $M$ \emph{akzeptierte Sprache}:
    \begin{multline*}
        L(M) = \{w \in \Sigma^*\; |\; \text{die Berechnung von}\\
        \text{$M$ auf $w$ endet in einer}\\
        \text{Endkonfiguration $(q, \lambda)$ mit $q \in F$}\}
    \end{multline*}
    \item $L_\textrm{EA} = \{L(M)\; |\; \text{$M$ ist ein EA\}}$ ist die Klasse der regulären Sprachen
    \begin{itemize}
        \item[\hookrightarrow] $L \in L_\textrm{EA} \iff L\ \text{ist regulär}$
    \end{itemize}
    \item $(q, w) \sststile{M}{*} (p, u) \iff (q = p \land w = u)$ oder man kommt in $k \in \mathbb{N}^+$ Schritten von $(q, w)$ nach $(p, u)$
    \item $\hat{\delta}(q, \lambda) = q$
    \item $\hat{\delta}(q, wa) = \delta(\hat{\delta}(q, w), a)\; a \in \Sigma,\, w \in \Sigma^*,\, q \in Q$\quad(\say{In welchem Zustand enden wir, wenn wir wa als Eingabe haben?})
    \item $\textrm{Kl}[p] = \{w \in \sigma^*\; |\; \hat{\delta}(q_0, w) = p\} = \{w \in \Sigma^*\; |\; (q_0, w) \sststile{M}{*} (p, \lambda)\}$
    \item $\textrm{Kl}[q_1] \cup \dots \cup \textrm{Kl}[q_n] = \Sigma^* \;\land\; \textrm{Kl}[q_i] \cap \textrm{Kl}[q_j] = \emptyset,\; i \neq j$
\end{itemize}
\subsection{Beweise der Nichtexistenz}
\begin{mainbox}{Lemma 3.3}
    Sei $A = (Q, \Sigma, \delta_A, q_0, F)$ ein EA. Seien $x, y \in \Sigma^*,\, x \neq y$, so dass
    \[(q_0, x) \sststile{A}{*} (p, \lambda)\ \text{und}\ (q_0, y) \sststile{A}{*} (p, \lambda)\]
    für ein $p \in Q$ (also $\hat{\delta}_A(q_o, x) = \hat{\delta}_A(q_0, y) = p\quad (x, y \in \textrm{Kl}[p])$). Dann existiert für jedes $z \in \Sigma^*$ ein $r \in Q$, so dass $xz$ und $yz \in \textrm{Kl}[r]$, also gilt insbesondere
    \[xz \in L(A) \iff yz \in L(A) \text{.}\]
\end{mainbox}
\begin{mainbox}{Pumping-Lemma für reguläre Sprachen}
    Sei $L$ regulär. Dann existiert eine Konstante $n_0 \in \mathbb{N}$, so dass sich jedes Wort $w \in \Sigma^*$ mit $|w| \geq n_0$ in drei Teile $y$, $x$ und $z$ zerlegen lässt, das heisst $w = yxz$, wobei
    \begin{enumerate}
        \item $|yx| \leq n_0$,
        \item $|x| \geq q$ und
        \item entweder $\{yx^kz \;|\; k \in \mathbb{N}\} \subseteq L$ oder $\{yx^kz \;|\; k \in \mathbb{N}\} \cap L = \emptyset$.
    \end{enumerate}
\end{mainbox}
\begin{mainbox}{Satz 3.1 (Präfixsprache)}
    Sei $L \subseteq (\Sigma_\textrm{bool})^*$ eine reguläre Sprache. Sei $L_x = \{y \in (\Sigma_\textrm{bool})^* \;|\; xy \in L\}$ für jedes $x \in (\Sigma_\textrm{bool})^*$. Dann existiert eine Konstante $\textrm{const}$, so dass für alle $x, y \in (\Sigma_\textrm{bool})^*$
    \[K(y) \leq \lceil \log_2 (n + 1) \rceil + \textrm{const} \text{,}\]
    falls $y$ das $n$-te Wort in der Sprache $L_x$ ist.
\end{mainbox}
\begin{mainbox}{Nichtdeterministischer endlicher Automat (NEA)}
    \begin{itemize}
        \item Wie ein EA, nur dass $\delta:\: Q \times \Sigma \to \mathcal{P}(Q)$
        \item $L(M) = \{w \in \Sigma^* \;|\; (q_0, w) \sststile{M}{*} (p, \lambda)\ \text{für ein}\ p \in F\}$
    \end{itemize}
\end{mainbox}

\section{Turingmaschinen}
% Kapitel 4
\begin{mainbox}{Turingmaschine}
    Sei $M = (Q, \Sigma, \Gamma, \delta, q_0, q_\textrm{accept}, q_\textrm{reject})$ eine Turingmaschine mit
    \begin{itemize}
        \item $Q$: \emph{Zustandsmenge},
        \item $\Sigma$: \emph{Eingabealphabet} ($\cent, \textvisiblespace \notin \Sigma$),
        \item $\Gamma$: \emph{Arbeitsalphabet},
        \item $\delta:\: (Q \setminus \{q_\textrm{accept}, q_\textrm{reject}\}) \times \Gamma \to Q \times \Gamma \times \{L, R, N\}$: \emph{Übergangsfunktion} (jetziger Zustand $\times$ jetziges Zeichen $\to$ nächster Zustand $\times$ neues Zeichen $\times$ Richtung des Lesekopfs).
    \end{itemize}
\end{mainbox}
\begin{itemize}
    \item $\textrm{Konf}(M) = \{\cent\} \cdot \Gamma^* \cdot Q \cdot (\Gamma^+ \cup Q) \cdot \{\cent\} \cdot \Gamma^+$
    \item Startkonfiguration: $q_0 \cent x$
    \item $L(M) = \{w \in \Sigma^* \;|\; q_0 \cent w \sststile{M}{*} y q_\textrm{accept} z,\;\text{für irgendwelche}\ y, z \in \Gamma^*\}$
    \item $M\ \text{hält immer} \iff M\ \text{landet immer in $q_\textrm{reject}$ oder $q_\textrm{accept}$}$
\end{itemize}

\section{Berechenbarkeit}
% Kapitel 5
\begin{itemize}
    \item $L_\textrm{EA} = \{L(M) \;|\; M\ \text{ist ein EA}\}$: Klasse der \emph{regulären Sprachen}.\\
    $\exists\ \text{EA}\ A:\: L(A) = L \iff L\ \text{mit EA erkennbar}$
    \item $L_\textrm{R} = \{L(M) \;|\; M\ \text{it eine TM, die immer hält}\}$: Klasse der \emph{rekursiven Sprachen}.\\
    $\exists\ \text{Alg.}\ A:\: L(A) = L \iff L\ \text{entscheidbar}$
    \item $L_\textrm{RE} = \{L(M) \;|\; M\ \text{ist eine TM}$: Klasse der \emph{rekursiv aufzählbaren Sprachen}.\\
    $\exists\ \text{TM}\ M:\: L(M) = L \iff L\ \text{erkennbar}$
\end{itemize}
\begin{align*}
    L_\textrm{diag} &= \{w \in (\Sigma_\textrm{bool})^* \;|\; w = w_i \land M_i\ \text{akz.}\ w_i\ \text{nicht}\}\\
    L_\textrm{U} &= \{\textrm{Kod}(M) \# w \;|\; w \in (\Sigma_\textrm{bool})^* \land M\ \text{akz.}\ w\}\\
    L_\textrm{H} &= \{\textrm{Kod}(M) \# x \;|\; x \in (\Sigma_\textrm{bool})^* \land M\ \text{hält auf}\ x\}\\
    L_\textrm{empty} &= \{\textrm{Kod}(M) \;|\; L(M) = \emptyset\}\\
    L_{\textrm{H}, \lambda} &= \{\textrm{Kod}(M) \;|\; M\ \text{hält auf}\ \lambda\}
\end{align*}
\begin{align*}
    L_\textrm{diag}^\complement, L_\textrm{empty}^\complement, L_\textrm{U}, L_\textrm{H}, L_{\textrm{H}, \lambda} &\in L_\textrm{RE} \setminus L_\textrm{R}\\
    L_\textrm{diag}, L_\textrm{empty}, L_\textrm{U}^\complement, L_\textrm{H}^\complement, L_{\textrm{H}, \lambda}^\complement &\in L_\textrm{RE}^\complement
\end{align*}
Bemerkung: $L, L^\complement \in L_\textrm{RE} \setminus L_\textrm{R}$ ist nicht möglich.
\begin{mainbox}{EE-Reduktion}
    Seien $L_1 \subseteq \Sigma_1^*,\, L_2 \subseteq \Sigma_2^*$ zwei Sprachen. Wir sagen, dass $L_1$ auf $L_2$ \emph{EE-reduzierbar} ist, \emph{$L_1 \leq_\textrm{EE} L_2$}, wenn eine TM $M$ existiert, die eine Abbildung $f_M:\: \Sigma_1^* \to \Sigma_2^*$ mit der Eigenschaft
    \[x \in L_1 \iff f_M(x) \in L_2\]
    für alle $x \in \Sigma_1^*$ berechnet. Wir sagen auch, dass die TM $M$ die Sprache $L_1$ auf die Sprache $L_2$ \emph{reduziert}.
\end{mainbox}
Es muss gelten:
\begin{enumerate}
    \item $F$ terminiert immer
    \item $F$ berechnet $f_M:\: \Sigma_1^* \to \Sigma_2^*$
    \item $x \in L_1 \iff f_M(x) \in L_2$
\end{enumerate}
1) und 2) gelten oft trivialerweise, 3) muss aber gezeigt werden!
Tricks:
\begin{itemize}
    \item Transitionen von TM umleiten (z.B. $q_\textrm{reject}$ nach Endlosschleife)
    \item TM konstuieren, die unabhängig von Eingabe immer dasselbe tut (z.B. TM die immer alles akzeptiert)
\end{itemize}
\begin{mainbox}{R-Reduktion}
    Seien $L_1 \subseteq \Sigma_1^*$ und $L_2 \subseteq \Sigma_2^*$ zwei Sprachen. Wir sagen, dass $L_1$ auf $L_2$ \emph{rekursiv reduzierbar} ist, $L_1 \leq_\textrm{R} L_2$, falls
    \[L_2 \in \mathcal{L}_R \implies L_1 \in \mathcal{L}_R \text{.}\]
\end{mainbox}
Aus $L_1 \leq_{EE} L_2$ folgt direkt $L_1 \leq_R L_2$.\\
Bemerkung: $L \in \mathcal{L}_R \iff L^\complement \in \mathcal{L}_R$

\section{Komplexitätstheorie}
% Kapitel 6
% \begin{subbox}{Zeitkomplexität}
%     Sei $M$ eine Mehrband-Turingmaschine, die immer hält. Sei $\Sigma$ das Eingabealphabet von $M$. Sei $x \in \Sigma^*$ und sei $D = C_1, C_2, \dots, C_k$ die Berechnung von $M$ auf $x$. Dann ist die \emph{Zeitkomplexität $\textrm{Time}_M(x)$ der Berechnung von $M$ auf $x$} definiert durch
%     \[\textrm{Time}_M(x) = k - 1 \text{,}\]
%     also durch die Anzahl der Berechnungsschritte in $D$.\\
%     Die \emph{Zeitkomplexität von $M$} ist die Funktion $\textrm{Time}_M:\: \mathbb{N} \to \mathbb{N}$, definiert durch
%     \[\textrm{Time}_M(n) = \max\{\textrm{Time}_M(x) \;|\; x \in \Sigma^n\} \text{.}\]
% \end{subbox}
% \begin{subbox}{Speicherplatzkomplexität}
    
% \end{subbox}
\subsection{Deterministische Komplexitätsmasse}
Für alle Funktionen $f, g$ von $\mathbb{N}$ nach $\mathbb{R}^+$ definieren wir
\begin{align*}
    \textrm{TIME}(f) &= \{L(B) \;|\; B\ \text{ist MTM mit}\ \textrm{Time}_B(n) \in \mathcal{O}(f(n))\}\\
    \textrm{SPACE}(g) &= \{L(A) \;|\; A\ \text{ist MTM mit}\ \textrm{Space}_A(n) \in \mathcal{O}(g(n))\}\\
    \textrm{DLOG} &= \textrm{SPACE}(\log_2 n)\\
    \textrm{P} &= \bigcup_{c \in \mathbb{N}} \textrm{TIME}(n^c)\\
    \textrm{PSPACE} &= \bigcup_{c \in \mathbb{N}} \textrm{SPACE}(n^c)\\
    \textrm{EXPTIME} &= \bigcup_{d \in \mathbb{N}} \textrm{TIME}(2^{n^d})
\end{align*}
\begin{itemize}
    \item $\textrm{TIME}(t(n)) \subseteq \textrm{SPACE}(t(n)) \implies \textrm{P} \subseteq \textrm{PSPACE}$
    \item $\textrm{DLOG} \subseteq \textrm{P} \subseteq \textrm{PSPACE} \subseteq \textrm{EXPTIME}$
\end{itemize}
\begin{subbox}{Platzkonstruierbarkeit}
    Eine Funktion $s:\: \mathbb{N} \to \mathbb{N}$ heisst \emph{platzkonstruierbar}, falls eine 1-Band-TM $M$ existiert, so dass
    \begin{enumerate}
        \item $\textrm{Space}_M(n) \leq s(n)$ für alle $n \in \mathbb{N}$ und
        \item für jede Eingabe $0^n,\; n \in \mathbb{N}$ generiert $M$ das Wort $0^{s(n)}$ auf ihrem Arbeitsband und hält in $q_\textrm{accept}$.
    \end{enumerate}
\end{subbox}
\begin{subbox}{Zeitkonstruierbarkeit}
    Eine Funktion $t:\: \mathbb{N} \to \mathbb{N}$ heisst \emph{zeitkonstruierbar}, falls eine MTM $A$ existiert, so dass
    \begin{enumerate}
        \item $\textrm{Time}_A(n) \in \mathcal{O}(t(n))$ und
        \item für jede Eingabe $0^n,\; n \in \mathbb{N}$ generiert $A$ das Wort $0^{t(n)}$ auf dem ersten Arbeitsband und hält in $q_\textrm{accept}$.
    \end{enumerate}
\end{subbox}
\subsection{Nichtdeterministische Komplexitätsmasse}
Für alle Funktionen $f, g$ von $\mathbb{N}$ nach $\mathbb{R}^+$ definieren wir
\begin{align*}
    \textrm{NTIME}(f) &= \{L(M) \;|\; M\ \text{ist NMTM mit}\ \textrm{Time}_M(n) \in \mathcal{O}(f(n))\}\\
    \textrm{NSPACE}(g) &= \{L(M) \;|\; M\ \text{ist NMTM mit}\ \textrm{Space}_M(n) \in \mathcal{O}(g(n))\}\\
    \textrm{NLOG} &= \textrm{NSPACE}(\log_2 n)\\
    \textrm{NP} &= \bigcup_{c \in \mathbb{N}} \textrm{NTIME}(n^c)\\
    \textrm{NPSPACE} &= \bigcup_{c \in \mathbb{N}} \textrm{NSPACE}(n^c)
\end{align*}
\begin{itemize}
    \item $\textrm{DLOG} \subseteq \textrm{NLOG} \subseteq \textrm{P} \subseteq \textrm{NP} \subseteq \textrm{PSPACE} \subseteq \textrm{EXPTIME}$
\end{itemize}
\begin{subbox}{Platzkonstruierbarkeit}
    Eine Funktion $s:\: \mathbb{N} \to \mathbb{N}$ heisst \emph{platzkonstruierbar}, falls eine 1-Band-TM $M$ existiert, so dass
    \begin{enumerate}
        \item $\textrm{Space}_M(n) \leq s(n)$ für alle $n \in \mathbb{N}$ und
        \item für jede Eingabe $0^n,\; n \in \mathbb{N}$ generiert $M$ das Wort $0^{s(n)}$ auf ihrem Arbeitsband und hält in $q_\textrm{accept}$.
    \end{enumerate}
\end{subbox}
\begin{subbox}{Zeitkonstruierbarkeit}
    Eine Funktion $t:\: \mathbb{N} \to \mathbb{N}$ heisst \emph{zeitkonstruierbar}, falls eine MTM $A$ existiert, so dass
    \begin{enumerate}
        \item $\textrm{Time}_A(n) \in \mathcal{O}(t(n))$ und
        \item für jede Eingabe $0^n,\; n \in \mathbb{N}$ generiert $A$ das Wort $0^{t(n)}$ auf dem ersten Arbeitsband und hält in $q_\textrm{accept}$.
    \end{enumerate}
\end{subbox}
\begin{subbox}{Polynomielle Reduktion}
    Seien $L_1 \subseteq \Sigma_1^*$ und $L_2 \subseteq \Sigma_2^*$ zwei Sprachen. Wir sagen, dass \emph{$L_1$ polynomiell reduzierbar auf $L_2$ ist, $L_1 \leq_\textrm{p}$}, falls eine polynomielle TM $A$ existiert, die für jedes Wort $x \in \Sigma_1^*$ ein Wort $A(x) \in \Sigma_2^*$ berechnet, so dass
    \[x \in L_1 \iff A(X) \in L_2 \text{.}\]
    $A$ wird eine \emph{polynomielle Reduktion} von $L_1$ auf $L_2$ genannt.
\end{subbox}
\begin{mainbox}{NP-schwer}
    Eine Sprache $L$ ist \emph{NP-schwer}, falls für alle Sprachen $L' \in \textrm{NP}$ gilt $L' \leq_\textrm{p} L$.
\end{mainbox}
\begin{mainbox}{NP-vollständig}
    Eine Sprach $L$ ist \emph{NP-vollständig}, falls
    \begin{enumerate}
        \item $L \in \textrm{NP}$ und
        \item $L$ ist NP-schwer.
    \end{enumerate}
\end{mainbox}
\begin{align*}
    \textrm{SAT} &= \{\Phi \;|\; \Phi\ \text{ist erfüllbare Formel in KNF}\}\\
    \textrm{CLIQUE} &= \{(G, k) \;|\; G\ \text{enthält eine $k$-Clique}\}\\
    \textrm{VC} &= \{(G, k) \;|\; G\ \text{hat Knotenüberdeckung höchstens $k$}\}
\end{align*}

\section{Grammatiken und Chomsky-Hierarchie}
% Kapitel 10
\begin{mainbox}{Grammatik}
    Eine Grammatik $G$ ist ein 4-Tupel $G = (\Sigma_N, \Sigma_T, P, S)$, wobei:
    \begin{enumerate}
        \item $\Sigma_N$ ist \emph{Nichtterminalalphabet}.
        \item $\Sigma_T$ ist \emph{Terminalalphabet}.\\
        Es gilt
        \[\Sigma_N \cap \Sigma_T = \emptyset \text{.}\]
        \item $S \in \Sigma_N$ heisst \emph{Startsymbol}.
        \item $P$ ist endliche Teilmenge von $\Sigma^* \Sigma_N \Sigma^* \times \Sigma^*$ für $\Sigma = \Sigma_N \cup \Sigma_T$, genannt die \emph{Menge der Ableitunsregeln von $G$}.
    \end{enumerate}
\end{mainbox}
\begin{subbox}{Chomsky-Hierarchie}
    Sei $G = (\Sigma_N, \Sigma_T, P, S)$ eine Grammatik.
    \begin{enumerate}
        \item $G$ heisst \emph{Typ-0-Grammatik}.
        \item $G$ heisst \emph{kontextsensitiv} oder \emph{Typ-1-Grammatik}, falls für alle Paare $(\alpha, \beta) \in P$
        \[|\alpha| \leq |\beta|\]
        gilt.
        \item $G$ heisst \emph{kontextfrei} oder \emph{Typ-2-Grammatik}, falls für alle Regeln $(\alpha, \beta) \in P$
        \[\alpha \in \Sigma_N\ \text{und}\ \beta \in (\Sigma_N \cup \Sigma_T)^*\]
        gilt.
        \item $G$ heisst \emph{regulär} oder \emph{Typ-3-Grammatik}, falls für alle Regeln $(\alpha, \beta) \in P$
        \[\alpha \in \Sigma_N\ \text{und}\ \beta \in \Sigma_T^* \cdot \Sigma_N \cup \Sigma_T^*\]
        gilt.
    \end{enumerate}
\end{subbox}
\begin{subbox}{Normierte reguläre Grammatik}
    Eine reguläre Grammatik heisst \emph{normiert}, wenn alle Regeln der Grammatik nur eine der folgenden drei Formen haben:
    \begin{enumerate}
        \item $S \to \lambda$, wobei $S$ das Startsymbol ist,
        \item $A \to a$ für $A \in \Sigma_N$ und $a \in \Sigma_T$, und
        \item $B \to bC$ für $B, C \in \Sigma_N$ und $b \in \Sigma_T$.
    \end{enumerate}
\end{subbox}
\begin{subbox}{Pumping-Lemma für kontextfreie Sprachen}
    Sei $L$ kontextfrei. Dann existiert eine nur von $L$ abhängige Konstante $n_L$, so dass für alle Wörter $z \in L$ mit $|z| \geq n_L$ eine Zerlegung
    \[z = uvwxy\]
    von $z$ existiert, so dass
    \begin{enumerate}
        \item $|vx| \geq 1$,
        \item $|vwx| \leq n_L$ und
        \item $\{uv^iwx^iy \;|\; i \in \mathbb{N} \subseteq L$.
    \end{enumerate}
\end{subbox}

\end{document}
