% Basic stuff
\documentclass[a4paper,10pt]{article}
\usepackage[nswissgerman]{babel}

% 3 column landscape layout with fewer margins
\usepackage[landscape, left=0.75cm, top=1cm, right=0.75cm, bottom=1.5cm, footskip=15pt]{geometry}
\usepackage{flowfram}
\ffvadjustfalse
\setlength{\columnsep}{1cm}
\Ncolumn{3}

% define nice looking boxes
\usepackage[many]{tcolorbox}

% a base set, that is then customised
\tcbset {
	base/.style={
		boxrule=0mm,
		leftrule=1mm,
		left=1.75mm,
		arc=0mm, 
		fonttitle=\bfseries, 
		colbacktitle=black!10!white, 
		coltitle=black, 
		toptitle=0.75mm, 
		bottomtitle=0.25mm,
		title={#1}
	}
}

\definecolor{brandblue}{rgb}{0.34, 0.7, 1}
\newtcolorbox{mainbox}[1]{
	colframe=brandblue, 
	base={#1}
}

\newtcolorbox{subbox}[1]{
	colframe=black!20!white,
	base={#1}
}

% Mathematical typesetting & symbols
\usepackage{amsthm, mathtools, amssymb} 
\usepackage{marvosym, wasysym}
\allowdisplaybreaks

% Tables
\usepackage{tabularx, multirow}
\usepackage{makecell}
\usepackage{booktabs}
\renewcommand*{\arraystretch}{2}

% Make enumerations more compact
\usepackage{enumitem}
\setitemize{itemsep=0.5pt}
\setenumerate{itemsep=0.75pt}

% To include sketches & PDFs
\usepackage{graphicx}

% For hyperlinks
\usepackage{hyperref}
\hypersetup{
	colorlinks=true
}

% Metadata
\title{Cheatsheet\\ Wahrscheinlichkeit \& Statistik}
\author{Julian Steinmann}
\date{Sommer 2022}

% Math helper stuff
\def\limn{\lim_{n\to \infty}}
\def\limxo{\lim_{x\to 0}}
\def\limxi{\lim_{x\to\infty}}
\def\limxn{\lim_{x\to-\infty}}
\def\sumk{\sum_{k=1}^\infty}
\def\sumn{\sum_{n=0}^\infty}
\def\R{\mathbb{R}}
\def\P{\mathbb{P}}
\def\F{\mathcal{F}}
\def\E{\mathbb{E}}
\DeclareMathOperator{\Var}{\text{Var}}


\begin{document}

\maketitle

% Not actually an abstract. We abuse it for licencing
\renewcommand{\abstractname}{Lizenz}
\begin{abstract}
	Dieses Dokument ist unter CC BY-SA 4.0 lizenziert. Es darf verbreitet oder verändert werden, solange der Urheber und die Lizenz erhalten bleibt.

	\begin{center}
		Der \LaTeX-Quelltext ist verfügbar auf \\ \href{https://github.com/XYQuadrat/eth-cheatsheets}{github.com/XYQuadrat/eth-cheatsheets}.
	\end{center}
\end{abstract}


\section{Grundbegriffe}
\begin{subbox}{Definition Wahrscheinlichkeitraum}
	Ein Wahrscheinlichkeitraum ist ein Tupel \((\Omega, \F, \P)\):
	\begin{itemize}
		\item Die Menge \(\Omega\) nenen wir \textbf{Grundraum}. Ein \(\omega \in \Omega\) nennen wir Elementarereignis.
		\item \(\F \subseteq \P(\Omega)\) ist eine \textbf{Sigma-Algebra}.
		\item \(\P\) ist ein \textbf{Wahrscheinlichkeitsmass} definiert auf \((\Omega, \F)\).
	\end{itemize}
	Dabei ist \(A \subseteq \Omega\) ein Ereignis.
\end{subbox}

\subsection{Sigma-Algebra}
Eine Sigma-Algebra ist ein Subset \(\F \subseteq \mathcal{P}(\Omega)\) mit den folgenden Eigenschaften:
\begin{enumerate}
	\item \(\Omega \in \F\)
	\item \(A \in \F \implies A^\complement \in \F\)
	\item \(A_1, A_2, \ldots \in \F \implies \bigcup_{i=1}^\infty A_i \in \F\)
	\item \(\varnothing \in \F\)
	\item \(A_1, A_2, \ldots \in \F \implies \bigcap_{i=1}^\infty A_i \in \F\)
	\item \(A, B \in \F \implies A \cup B \in \F\)
	\item \(A, B \in \F \implies A \cap B \in \F\)
\end{enumerate}
Nützlich ist ausserdem die De-Morgan-Regel: \[(\cup_{i=1}^\infty A_i)^\complement = \cap_{i=1}^\infty(A_i)^\complement\]

\subsection{Wahrscheinlichkeitsmass}
Ein Wahrscheinlichkeitsmass \(\P\) ist eine Abbildung
\begin{align*}
	\P: \  & \F \mapsto \left[0,1\right] \\
				 & A \mapsto \P[A]
\end{align*}
mit den Eigenschaften
\begin{enumerate}
	\item \(\P[\Omega] = 1\)
	\item \(\P [A] = \sum_{i=1}^\infty P[A_i]\), falls \(A = \bigsqcup_{i=1}^\infty A_i\)
	\item \(\P[\varnothing] = 0\)
	\item \(\P[A^\complement] = 1 - \P[A]\)
	\item \(\P[A \cup B] = \P[A] + P[B] - \P[A\cap B]\)
	\item \(A \subseteq B \implies \P[A] \le \P[B]\) (Monotonie)
	\item \(\P[\bigcup_{i=1}^\infty A_i] \le \sum_{i=1}^\infty \P[A_i]\) (Union Bound)
\end{enumerate}

Wenn \(A_1, \ldots A_n\) paarweise disjunkt sind, gilt:
\[\P[A_1 \cup \ldots \cup A_n] = \P[A_1] + \ldots + \P[A_n]\]

\subsection{Bedingte Wahrscheinlichkeiten}
Sei \((\Omega, \F, \P)\) ein Wahrscheinlichkeitsraum mit \(A, B \in \F\) und \(\P[B] > 0\). Die bedingte Wahrscheinlichkeit von \(A\) gegeben \(B\) ist definiert als:
\[\P[A|B] = \frac{\P[A\cap B]}{\P[B]}\]
\begin{subbox}{Totale Wahrscheinlichkeit}
	Sei \(B_1, \ldots, B_n \in \F\) eine Partition von \(\Omega\) mit \(\P[B_i] > 0\) für alle \(1 \le i \le n\). Dann gilt:
	\[\forall A \in \F \quad \P[A] = \sum_{i=1}^{n} \P[A|B_i] \ \P[B_i]\]
\end{subbox}
\begin{mainbox}{Satz von Bayes}
	Sei \(B_1, \ldots, B_n \in \F\) eine Partition von \(\Omega\) mit \(\P[B_i] > 0\) für jedes \(i\). Für jeden Event \(A\) mit \(\P[A] > 0\) gilt:
	\[\forall i = 1,\ldots, n \quad \P[B_i|A] = \frac{\P[A|B_i] \ \P[B_i]}{\sum_{k=1}^n \P[A|B_k] \ \P[B_k]}\]
\end{mainbox}

\subsection{Unabhängigkeit}
\begin{mainbox}{Definition Unabhängigkeit}
	Zwei Ereignisse \(A, B\in \F\) sind unabhängig, falls gilt:
	\[\P[A\cap B] = \P[A] \cdot \P[B]\]
	Alternative Definitionen sind:
	\begin{align*}
		\P[A|B] & = \P[A] \\
		\P[B|A] & = \P[B]
	\end{align*}
\end{mainbox}
\begin{itemize}
	\item Falls \(\P[A] \in \{0,1\}\), dann ist \(A\) zu jedem Ereignis unabhängig.
	\item Wenn ein Ereignis \(A\) unabhängig zu sich selbst ist, dann folgt \(\P[A] \in \{0,1\}\).
	\item Wenn \(A, B\) unabhängig sind, dann sind auch \(A, B^\complement\) unabhängig.
\end{itemize}
\subsubsection*{Unabhängigkeit für mehrere Ereignisse}
Seien \(A_1, \ldots, A_n \in F\), dann sind die Ereignisse unabhängig, falls gilt:
\[\forall I \subseteq \{1, \ldots, n\} \quad \P[\bigcap_{i\in I}A_i] = \prod_{i\in I} \P[A_i]\]

\section{Zufallsvariablen}
Eine Zufallsvariable ist eine Abbildung \(X: \Omega \mapsto \R\) mit
\[\forall x \in \R \quad \{\omega \in \Omega \mid X(\omega) \leq x\} \in \F\]
Dabei lassen wir das \(\omega\) oft weg und schreiben nur \(X\).

\subsection{Verteilungsfunktion}
\begin{mainbox}{Definition Verteilungsfunktion}
	Die Verteilungsfunktion ist die Abbildung \(F_X: \ \R \mapsto [0,1]\) definiert durch
	\[\forall x \in \R \quad F_X(x) = \P[X \le x]\]
\end{mainbox}
Die Verteilungsfunktion hat folgende Eigenschaften:
\begin{enumerate}
	\item \(a < b \implies \P[a < X \le b] = F_X(b) - F_X(a)\)
	\item \(F\) ist monoton wachsend
	\item \(F\) ist rechtsstetig, d.h. \(\lim_{t \to 0} F_{x+t} = F(x)\)
	\item \(\lim_{x\to - \infty} F_X(x) = 0\) und \(\lim_{x\to \infty} F_X(x) = 1\)
\end{enumerate}
\subsubsection*{Linksstetigkeit}
Die Definition der Linksstetigkeit ist
\[F(x-) = \lim_{t\to 0} F(x-t)\]
Es gilt allerdings \textit{nicht} immer \(F(x-) = F(x)\), d.h. nicht jede Verteilungsfunktion ist linksstetig. Allerdings gilt:
\[\forall x \in \R \quad F(x) - F(x-) = \P[X=x]\]
Daraus lässt sich für stetige ZV \(P[X=x] = 0\) folgern.
\subsection{Unabhängigkeit}
Die Zufallsvariablen \(X_1, \ldots, X_n\) sind unabhängig, falls:
\begin{align*}
	\forall x_1, \ldots, x_n \in \R: \quad & \P[X_1 \le x_1, \ldots, X_n \le x_n] =             \\
																				 & \P[X_1 \le x_1] \cdot \ldots \cdot \P[X_n \le x_n]
\end{align*}
Eine Folge von Zufallsvariablen \(X_1, X_2, \ldots\) ist unabhängig, falls \(
\forall n \ X_1, \ldots X_n\) unabhängig sind. Sie ist zusätzlich identisch verteilt (uiv.), falls ausserdem \(\forall i,j \quad F_{X_i} = F_{X_j}\) gilt.
\subsection{Transformationen}
Sei \(\varphi: \ \R \mapsto \R\) und \(X\) eine Zufallsvariable, so ist
\[\varphi(X) = \varphi \circ X\]
auch eine Zufallsvariable. Seien \(X_1, \ldots, X_n\) ZVs mit \(\phi: \R^n \mapsto \R\), so ist
\[\phi(X_1, \ldots, X_n) = \phi \circ (X_1, \ldots, X_n)\]
ebenfalls eine Zufallsvariable.

\subsection{Konstruktion einer Zufallsvariable}
Gegeben eine Verteilungsfunktion \(F_X\) wollen wir die entsprechende ZV \(X\) konstruieren.
\begin{subbox}{Kolmogorov-Theorem}
	Es existiert \((\Omega, \F,\P)\) und ZVs \( X_1, X_2, \ldots\), sodass \(X_1, X_2, \ldots\) uiv. Bernoullivariablen mit \(p = 0.5\) sind.
\end{subbox}

Sei \(X_1, X_2,\ldots \sim \text{Ber}(1/2)\) eine unendliche Folge, dann ist
\[U = \sum_{n = 1}^\infty 2^{-n}\cdot X_n\]
gleichverteilt auf \([0,1]\).

Aufgrund der Eigenschaften der Verteilungsfunktion \(F\) wissen wir, dass eine eindeutige Inverse $F^{-1}$ existiert. Wir können die generalisierte Inverse definieren als: \[\forall \alpha \in [0,1] \quad F^{-1}(\alpha) = \inf \{x \in \R \mid F(x) \geq \alpha\}\]

Sei nun \(F\) eine Verteilungsfunktion und \(U\) eine gleichverteilte ZV in \([0,1]\). Dann besitzt \(X = F^{-1}(U)\) genau die Verteilungsfunktion \(F_X = F\).

\begin{subbox}{Fast sichere Ereignisse}
	Ein Ereignis \(A \in \F\) tritt fast sicher (f.s) ein, falls \(\P[A] = 1\). Seien \(X, Y\) ZV, so schreiben wir \(X \le Y \text{ f.s.} \iff \P[X\le Y] = 1\).
\end{subbox}

\subsection{Diskrete Zufallsvariablen}
\begin{subbox}{Definition diskrete ZV}
	Eine ZV \(X\) heisst diskret, falls \(\exists W \subset \R\) endlich oder abzählbar, so dass \(X \in W\) f.s. Falls \(\Omega\) endlich oder abzählbar ist, dann ist \(X\) immer diskret.
\end{subbox}
\noindent Die Verteilungsfunktion einer diskreten ZV ist:
\[(p(x))_{x \in W} \text{ wobei } \sum_{x\in W} p(x) = 1\]
Die Gewichtsfunktion einer diskreten ZV ist:
\[\forall x \in W \quad p(x) = \P[X=x]\]

\subsection{Diskrete Verteilungen}
\textbf{Bernoulli-Verteilung} (\(X \sim \text{Ber}(p)\)): Hat nur die Ereignisse \(\{0,1\}\). Sie ist definiert als
\[\P[X=0] = 1-p \text{ und } \P[X=1]=p\]

\noindent \textbf{Binomialverteilung} (\(X \sim \text{Bin}(n,p)\)): Die Wiederholung von Bernoulli-Experimenten. Sie ist definiert als
\[\forall k \in \{0, \ldots, n\} \quad \P[X=k] = \binom{n}{k} \cdot p^k \cdot (1-p)^{n-k}\]

\noindent \textbf{Geometrische Verteilung} (\(X \sim \text{Geom}(p)\)): Beschreibt das erste Auftreten eines Erfolges. Sie ist definiert als
\[\forall k \in \mathbb{N} - \{0\} \quad \P[X=k]=(1-p)^{k-1}\cdot p\]

\noindent \textbf{Poisson-Verteilung} (\(X \sim \text{Poisson}(\lambda)\)): Annäherung an die Binomialverteilung für grosse \(n\) und kleine \(p\) (d.h. rare Ereignisse). Sie ist definiert als
\[\forall k \in \mathbb{N}, \lambda > 0 \quad \P[X=k]=\frac{\lambda^k}{k!}\cdot e^{-\lambda}\]

\subsection{Stetige Zufallsvariablen}
\begin{subbox}{Definition stetige ZV}
	Eine ZV \(X\) heisst stetig, wenn ihre Verteilungsfunktion \(F_X\) wie folgt geschrieben werden kann:
	\[\forall x \in \R \quad F_X(x) = \int_{-\infty}^x f(y) \mathop{dy}\]
\end{subbox}
Hierbei ist \(f: \R \mapsto \R_+\) die Dichte von X. Für die Dichte gilt:
\[\int_{-\infty}^{+\infty}f(y) \mathop{dy} = 1\]
Es gelten folgende Eigenschaften:
\begin{enumerate}
	\item \(\P[a \le x \le b] = \P[a < x < b]\)
	\item \(\P[X=x] = 0\)
	\item \(\P[X \in [a,b]] = \P[X \in (a,b)]\)
\end{enumerate}

\subsection{Stetige Verteilungen}
\textbf{Gleichverteilung} (\(X \sim \mathcal{U}[a,b]\)): Jedes Ereignis hat die gleiche Wahrscheinlichkeit. Sie ist definiert als
\[f_{a,b}(x) = \begin{cases}
	0             & x \notin [a,b] \\
	\frac{1}{b-a} & x \in [a,b]
\end{cases}\]

\noindent \textbf{Exponentialverteilung} (\(X \sim \text{Exp}(\lambda)\)): Das stetige Gegenstück zur Geometrischen Verteilung. Sie ist definiert als
\[f_{a,b}(x) = \begin{cases}
	0                            & x < 0    \\
	\lambda \cdot e^{-\lambda x} & x \geq 0
\end{cases}\]
\begin{itemize}
	\item Wenn \(Y\sim \text{Exp}(\lambda)\), dann ist \(c \cdot Y \sim \text{Exp}(\frac{\lambda}{c})\)
	\item \(X_1, X_2 \sim \mathcal{N}(0,1)\) uiv. \(\implies \chi_2 = X_1^2 +X_2^2 \sim \text{Exp}(\frac{1}{2})\)
\end{itemize}

\noindent \textbf{Normalverteilung} (\(X \sim \mathcal{N}(m, \sigma^2)\)): Die wohl wichtigste Verteilung. Sie ist definiert als
\[f_{m, \sigma}(x) = \frac{1}{\sqrt{2 \pi \sigma^2}} \cdot e^{-\frac{(x-m)^2}{2 \sigma^2}}\]

\(X \sim \mathcal{N}(0,1)\) wird auch Standardnormalverteilung genannt. Für eine standardnormalverteilte ZV \(X\) gilt
\[Z = m +\sigma \cdot X \sim \mathcal{N}(m, \sigma^2)\]

\section{Erwartungswert}
\begin{mainbox}{Definition Erwartungswert}
	Sei \(X: \Omega \mapsto \R_+\) eine ZV mit nicht-negativen Werten. Dann ist
	\[\E[X] = \int_0^\infty 1- F_X(x) \mathop{dx}\]
	der Erwartungswert von \(X\).
\end{mainbox}
Wenn \(\E[|X|] < \infty\), dann ist der Erwartungswert definiert als
\[\E[X] = \E[X_+] - \E[X_-]\]

\subsection{Erwartungswert diskreter ZV}
Sei \(X\) eine diskrete ZV mit \(X \in W\) f.s. Sei \(\phi: \R \mapsto \R\) eine Abbildung. Falls die Summe wohldefiniert ist, gilt
\[\E[\phi(X)] = \sum_{x\in W} \phi(x)\cdot \P[X=x]\]
Sei \(\phi = \) id, gilt
\[
	\E[X] = \sum_{x\in W} x \cdot \P[X=x]
.\] 

\subsection{Erwartungswert stetiger ZV}
Sei \(X\) eine stetige ZV mit Dichtefunktion \(f\). Sei \(\phi :\R\mapsto \R\) eine Abbildung, sodass \(\phi(x)\) eine Zufallsvariable ist. Sofern das Integral wohldefiniert ist, gilt
\[\E[\phi(X)] = \int_{-\infty}^{\infty}\phi(x)f(x) \mathop{dx}\]
Die Definition für \(\phi = \) id ist analog zum diskreten Fall.
\subsubsection{\texorpdfstring{Bestimmen der Dichte von \(f(X)\)}{Bestimmen der Dichte von f(X)}}
\begin{enumerate}
	\item Sei \(\phi: \R \mapsto \R\) stückweise beschränkt und stetig.
	\item \(\E[\phi(Y)] = \E[\phi(f(X))] = \E[\tau(X)]\)
	\item Dichte von \(X\) in vorheriger Gleichung einsetzen.
	\item Variablenwechsel \(u = f(x)\) \& Grenzen anpassen
\end{enumerate}

\subsection{Rechnen mit Erwartungswerten}
\begin{subbox}{Linearität des Erwartungswertes}
	Seien \(X,Z\) ZV mit \(\lambda \in \R\). Falls die Erwartungswerte wohldefiniert sind, gilt
	\begin{align*}
		\E[\lambda \cdot X] & = \lambda \cdot \E[X] \\
		\E[X + Y]           & = \E[X] + \E[Y]       \\
	\end{align*}
\end{subbox}
Falls zwei ZV \(X,Y\) unabhängig sind, gilt auch
\[\E[X\cdot Y] = \E[X] \cdot \E[Y]\]

Für Divison hingegen gilt:
\[\E[\frac{X}{Y}] = \E[X] \cdot \E[\frac{1}{Y}]\]
wobei \(\E[1/Y]\) separat berechnet werden muss.

\begin{subbox}{Alternativdefinition unabhängige ZV}
	Seien \(X_1, \ldots, X_n\) diskrete ZV. Dann sind folgende Aussagen äquivalent:
	\begin{enumerate}
		\item \(X_1, \ldots, X_n\) sind unabhängig.
		\item Für jedes \(\phi_1, \ldots, \phi_n: \R \mapsto\R\) beschränkt gilt:
	\end{enumerate}
\[\E[\phi_1(X_1)\cdot\ldots\cdot\phi_n(X_n)] = \E[\phi_1(X_1)] \cdot\ldots\cdot \E[\phi_n(X_n)]\]
\end{subbox}

\subsection{Extremwertformel}
Sei \(X\) eine diskrete ZV mit Werten in \(\mathbb{N}\). Dann gilt folgende Identität:
\[\E[X] = \sum_{n=1}^\infty \P[X\ge n]\]
Sei \(X\) eine stetige ZV mit \(X \ge 0\) f.s., dann gilt:
\[\E[X] = \int_0^\infty \P[X > x] \mathop{dx}\]

\subsection{Ungleichungen}
\begin{subbox}{Monotonie}
	Seien \(X, Y\) ZV sodass \(X \le Y\) f.s., dann gilt \(\E[X] \le \E[Y]\).
\end{subbox}

\begin{mainbox}{Markov-Ungleichung}
	Sei \(X\) eine ZV mit \(X \ge 0\) f.s., dann gilt für jedes \(a > 0\):
	\[\P[X\ge a] \le \frac{\E[X]}{a}\]
\end{mainbox}

\begin{subbox}{Jensen-Ungleichung}
	Sei \(X\) eine ZV und \(\phi : \R \mapsto \R\) eine konvexe Funktion, dann gilt:
	\[\phi(\E[X]) \le \E[\phi(X)]\]
\end{subbox}

\subsection{Varianz}
Sei \(X\) eine ZV sodass \(\E[X^2] < \infty\). Die Varianz von \(X\) ist definiert durch
\[\Var(X) = \sigma_X^2 = \E[(X-m)^2]\]
wobei \(m=\E[X]\). Dabei wird \(\sigma_X\) auch die Standardabweichung von \(X\) genannt und beschreibt die typische Distanz eines Wertes \(x\in X\) zu \(\E[X]\).

\begin{subbox}{Chebychev-Ungleichung}
	Wenn \(X\) eine ZV mit \(\E[X^2] < \infty\) ist, dann gilt für jedes \(a \ge 0\):
	\[\P[|X - \E[X]| \ge a] \le \frac{\sigma_X^2}{a^2}\]
\end{subbox}

\begin{enumerate}
	\item Wenn \(\E[X^2] < \infty\), dann \(\sigma_X^2 = \E[X^2] - \E[X]^2\).
	\item Wenn \(\E[X^2] < \infty\) und \(\lambda \in \R\), dann \(\sigma_{\lambda X}^2 = \lambda^2\sigma_X^2\).
	\item Wenn \(S = X_1 + \ldots + X_n\), wobei \(X_1, \ldots, X_n\) paarweise unabhängig sind, dann gilt \(\sigma_S^2 = \sigma_{X_1}^2 + \ldots + \sigma_{X_n}^2\).
\end{enumerate}

\subsection{Kovarianz}
Wir können mit der Kovarianz die Abhängigkeit von zwei Zufallsvariablen messen.
\begin{subbox}{Definition Kovarianz}
	Wenn \(X, Y\) zwei ZV mit \(\E[X^2] < \infty, \E[Y^2] < \infty\), dann ist die Kovarianz zwischen \(X, Y\) definiert als
	\[\text{Cov}(X,Y) = \E[X \cdot Y] - \E[X] \cdot \E[Y]\]
\end{subbox}
\begin{itemize}
	\item \(\text{Cov}(X,X) = \sigma_X^2\)
	\item X, Y unabhängig \(\implies \text{Cov}(X,Y) = 0\)
\end{itemize}

\section{Gemeinsame Verteilungen}
\subsection{Diskrete gemeinsame Verteilungen}
\begin{subbox}{Definition gemeinsame Verteilung}
	Seien \(X_1, \ldots, X_n\) diskrete Zufallsvariablen wobei \(X_i \in W_i\) f.s. für \(W_i \subset \R\). Die gemeinsame Verteilung (GV) von \(X_1, \ldots, X_n\) ist die Familie \(p = (p(x_1, \ldots, x_n))_{x_1 \in W_1, \ldots, x_n \in W_n}\) definiert durch
	\[p(x_1, \ldots, x_n) = \P[X_1 = x_1, \ldots, X_n = x_n]\]
\end{subbox}

Seien \(X_1,\ldots,X_n\) diskrete ZV mit \(X_i \in W_i\) f.s. für \(W_i \subset \R\) und \(\phi: \R^n \mapsto \R\), so ist \(Z = \phi(X_1, \ldots, X_n)\) eine diskrete ZV mit Werten in \(W = \phi(W_1 \times \ldots \times W_n)\) und folgender Verteilung:
\[\forall z \in W. \ \P[Z = z] = \sum_{\substack{\phi(x_1, \ldots, x_n) \\= z}} \P[X_1 = x_1, \ldots, X_n = x_n]\]

Seien \(X_1,\ldots,X_n\) diskrete ZV mit \(X_i \in W_i\) f.s. mit GV \(p\). Dann ist die Randverteilung \(\forall z \in W_i\):
\[\P[X_i = z] = \sum_{\substack{x_1, \ldots, x_{i-1}, \\x_{i+1},\ldots,x_n}} p(x_1, \ldots, x_{i-1}, z, x_{i+1},\ldots,x_n)\]

Seien \(X_1,\ldots,X_n\) diskrete ZV mit GV \(p\) und \(\phi : \R^n \mapsto \R\). Dann ist der Erwartungswert definiert als:
\[\E[\phi(X_1, \ldots, X_n)] = \sum_{x_1,\ldots,x_n} \phi(x_1,\ldots,x_n) \cdot p(x_1,\ldots,x_n)\]

Seien \(X_1,\ldots,X_n\) diskrete ZV mit GV \(p\), dann sind die folgenden Aussagen äquivalent:
\begin{enumerate}
	\item \(X_1,\ldots,X_n\) sind unabhängig
	\item Für alle \(x_1 \in W_1, \ldots, x_n \in W_n\) gilt:
		\[p(x_1,\ldots,x_n) = \P[X_1 = x_1] \cdot \ldots \cdot \P[X_n = x_n]\]
\end{enumerate}

Die gemeinsame Verteilung von \(X_1, \ldots, X_n\) erfüllt
\[\sum_{x_1\in W_1, \ldots, x_n \in W_n} p(x_1, \ldots, x_n) = 1\]

\subsection{Stetige gemeinsame Verteilungen}
\begin{subbox}{Definition gemeinsame Verteilung}
	Seien \(X_1, \ldots, X_n\) stetige ZV, so haben sie eine gemeinsame Verteilung, falls eine Funktion \(f: \R^n \mapsto \R_+\) existiert, die für jedes \(a_1, \ldots, a_n \in \R\) folgende Eigenschaft erfüllt:
	\begin{align*}
		\P[X_1 \le a_1, \ldots, X_n \le a_n] \\= \int_{-\infty}^{a_1} \cdots \int_{-\infty}^{a_n} f(x_1, \ldots, x_n) \mathop{dx_n} \ldots \mathop{dx_1}
	\end{align*}
	Dann ist \(f\) die \textbf{gemeinsame Dichte}.
\end{subbox}

Seien \(X_1, \ldots, X_n\) stetige ZV mit einer gemeinsamen Dichte \(f\) und \(\phi: \R^n \mapsto \R\). Dann ist der Erwartungswert definiert als
\begin{align*}
	\E[\phi(X_1, \ldots, X_n)] \\= \int_{-\infty}^\infty \cdots \int_{-\infty}^\infty \phi(x_1, \ldots, x_n) f(x_1, \ldots, x_n) \mathop{dx_n} \ldots \mathop{dx_1}
\end{align*}

Falls \(X_1, \ldots, X_n\) eine gemeinsame Dichte \(f\) besitzen, ist die Randverteilung 
\begin{align*}
	f_i(z) = \int_{x_1, \ldots, x_{i-1}, x_{i+1}, \ldots, x_n \in \R^{n-1}}\\
	f(x_1, \ldots, x_{i-1}, z, x_{i+1}, \ldots, x_n) \mathop{dx_n} \ \ldots \mathop{dx_1} \\
\end{align*}

Wenn \(X_1, \ldots, X_n\) stetige ZV mit Dichten \(f_1, \ldots, f_n\) sind, dann sind die folgenden Aussagen äquivalent:
\begin{enumerate}
	\item \(X_1, \ldots, X_n\) sind unabhängig
	\item \(X_1, \ldots, X_n\) sind stetig mit gemeinsamer Dichte
		\[f(x_1, \ldots, x_n) = f_1(x_1) \cdot \ldots \cdot f_n(x_n)\]
	\item Für alle \(\phi_1, \ldots, \phi_n: \R \mapsto \R\) gilt:
		\[\E[\phi_1 (x_1)\cdot \ldots\cdot (x_n)] = \E[\phi_1(x_1)] \cdot \ldots \cdot \E[\phi_n(x_n)]\]
\end{enumerate}

\section{Grenzwertsätze}
Sei \(X_1, X_2, \ldots\) eine unendliche Sequenz an uiv. ZV. Wir betrachten die Teilsumme \(S_n = X_1 + \ldots + X_n\).
\begin{mainbox}{Gesetz der grossen Zahlen}
	Sei \(\E[|X_1|] < \infty\) und \(m = \E[X_1]\), so gilt
	\[\lim_{n\to \infty} \frac{X_1 + \ldots + X_n}{n} = m \quad \text{f.s.}\]
\end{mainbox}
Da die ZV uiv. sind, gilt \(\E[|X_i|] < \infty\) und \(m = \E[X_i]\) auch für alle \(i\).

\subsubsection*{Konvergenz in Verteilung}
Seien \((X_n)_{n\in \mathbb{N}}\) und \(X\) ZV. Wir schreiben
\[X_n \approx X \quad \text{für } n \to \infty\]
falls \(\forall x \in \R\) gilt:
\[\lim_{n\to\infty} \P[X_n \le x] = \P[X \le x ]\]

\begin{mainbox}{Zentraler Grenzwertsatz}
	Sei \(\E[X^2_1] < \infty\) und wohldefiniert. Weiter sei \(m = \E[X_1]\) und \(\sigma^2 = \Var(X_1)\), so gilt:
	\begin{align*}
		\P\left[\frac{S_n - nm}{\sqrt{\sigma^2 n}} \leq a\right] \xrightarrow[n \to \infty]{} \Phi(a) = \frac{1}{\sqrt{2 \pi}} \int_{-\infty}^a e^{\frac{-x^2}{2}} \mathop{dx}
	\end{align*}
\end{mainbox}
Der zentrale Grenzwertsatz sagt aus, dass die Verteilung einer ZV
\[Z_n = \frac{S_n - nm}{\sqrt{\sigma^2 n}}\]
wie die Verteilung von \(\mathcal{N}(0,1)\) aussieht. Es gilt
\[Z_n \approx Z \quad \text{für } n\to \infty\]
wobei \(Z \sim \mathcal{N}(0,1)\). Für normalverteilte ZV \(X_1, \ldots, X_n\) ist \(Z_n\) immer standardnormalverteilt.

\section{Schätzer}
Wir treffen folgende Annahmen:
\begin{itemize}
	\item Parameterraum \(\theta \subset \R\)
	\item Familie von Wahrscheinlichkeitsmassen \((\P_\theta)_{\theta \in \Theta}\) auf \((\Omega, \F)\); für jedes Element im Parameterraum existiert ein Wahrscheinlichkeitsmodell
	\item Zufallsvariablen \(X_1, \ldots, X_n\) auf \((\Omega, \F)\)
	\item Wir nennen die Gesamtheit der beobachteten Daten \(x_1, \ldots, x_n\) oder der ZV \(X_1, \ldots, X_n\) Stichprobe
\end{itemize}
\begin{mainbox}{Definition Schätzer}
	Ein Schätzer ist eine Zufallsvariable \(T: \Omega \mapsto \R\) von der Form
	\[T = t(X_1, \ldots, X_n), \quad t: \R^n \mapsto \R\]
\end{mainbox}
Ein Schätzer \(T\) ist \textbf{erwartungstreu}, falls für alle \(\theta \in \Theta\) gilt:
\[\E_\theta [T] = \theta\]
Sei \(\theta \in \Theta\) und \(T\) ein Schätzer. Der \textbf{Bias} (erwartete Schätzfehler) von \(T\) im Modell \(\P_\theta\) ist definiert als:
\[\E_\theta[T]-\theta\]
Der mittlere quadratische Schätzfehler (MSE) von \(T\) im Modell \(\P_\theta\) ist definiert als:
\begin{align*}
	\text{MSE}_\theta[T] & = \E_\theta[(T-\theta)^2]                    \\
	\text{MSE}_\theta[T] & = \Var_\theta(T) + (\E_\theta[T] - \theta)^2
\end{align*}

\subsection{Maximum-Likelihood-Methode}
\subsubsection{Likelihood-Funktion, ML-Schätzer}
Die Likelihood-Funktion ist definiert als
\[L(x_1, \ldots, x_n; \theta) = \begin{cases}
	p(x_1, \ldots, x_n; \theta) \quad \text{(diskret)} \\
	f(x_1, \ldots, x_n; \theta) \quad \text{(stetig)}
\end{cases} \]

\noindent Für jedes \(x_1, \ldots, x_n \in W\) sei \(t_{ML}(x_1, \ldots, x_n)\) der Wert, welcher die Funktion \(\Theta \mapsto L(x_1, \ldots, x_n; \theta)\) maximiert. Ein Maximum-Likelihood-Schätzer ist dann definiert als
\[T_{ML} = t_{ML}(X_1, \ldots, X_n)\]

\subsection{Anwendung der Methode}
Die Maximum-Likelihood-Methode ist ein Weg, um systematisch einen Schätzer zu bestimmen.
\begin{enumerate}
	\item Gemeinsame Dichte/Verteilung der ZV finden
	\item Bestimme davon die Log-Likelihood-Funktion\\ \(f(\theta) := \ln(L(x_1, \ldots, x_n;\theta))\) 
	\item \(f(\theta)\) nach \(\theta\) ableiten
	\item Nullstelle von \(f'(\theta)\) finden
\end{enumerate}
Unter dem gefundenen \(\theta\) ist die Likelihood-Funktion maximal.

\section{Konfidenzintervalle}
\begin{mainbox}{Definition Konfidenzintervall}
	Sei \(\alpha \in [0,1]\). Ein Konfidenzintervall für \(\theta\) mit Niveau \(1 - \alpha\) ist ein Zufallsintervall \(I=[A,B]\), sodass gilt
	\[\forall \theta \in \Theta \quad \P_\theta[A\le \theta \le B] \ge 1- \alpha\]

	wobei \(A\) und \(B\) Zufallsvariablen der Form \(A = a(X_1, \ldots, X_n), B = b(X_1, \ldots, X_n)\) mit \(a,b: \R^n \to \R\) sind.
\end{mainbox}

Wenn wir einen Schätzer \(T = T_{ML} \sim \mathcal{N}(m, \frac{1}{n})\) haben, suchen wir ein Konfidenzintervall der Form
\[I = [T-c/\sqrt{n}, T+c/\sqrt{n}]\]
Hierbei gilt:
\begin{align*}
	\P_\theta[T-c/\sqrt{n} \le m \le T+c/\sqrt{n}] \\
	= \P_\theta[-c\le Z \le c]
\end{align*}
wobei \(Z = \sqrt{n}(T-m)\) ist.

\subsection{Häufige Fälle}
\subsubsection{\texorpdfstring{Normalverteilt - \(\mu\) unbekannt, \(\sigma^2\) bekannt}{Normalverteilt - μ unbekannt, σ² bekannt}}
Erwartungstreuer Schätzer: \(\overline{X}_n = \frac{1}{n} \sum_{i=1}^n X_i\)\\
Verteilung unter \(\P_\theta: \frac{\overline{X}_n - \theta_0}{\sqrt{\sigma^2/n}} \sim \mathcal{N}(0,1)\)
\begin{enumerate}
	\item Modell \(X_1, \ldots, X_n \sim \mathcal{N}(\theta, \sigma^2)\) uiv. unter \(\P_\theta\)
	\item Hypothesen \(H_0 : \theta = \theta_0\), z.B. \(H_A : \theta \ne \theta_0\)
	\item Test \(T = \frac{\overline{X}_n - \mu}{\sqrt{\sigma^2/n}} \sim \mathcal{N}(0,1)\)
	\item Verwerfungsbereich \(]-\infty, -c[ \ \cup \ ] c, \infty\) für \(c\ge 0\)
\end{enumerate}

\subsubsection{\texorpdfstring{Normalverteilt - \(\mu\), \(\sigma^2\) unbekannt}{Normalverteilt - μ, σ² unbekannt}}
Wir definieren \(\vec{\theta} = (\mu, \sigma^2)\) und den Varianz-Schätzer \(S^2 = \frac{1}{n-1}\sum_{i=1}^n (X_i - \overline{X}_n)^2\).
\begin{enumerate}
	\item Modell \(X_1, \ldots, X_n \sim \mathcal{N}(\theta, \sigma^2)\) uiv. unter \(\P_{\vec{\theta}}\)
	\item Test \(T = \frac{\overline{X}_n - \mu_0}{\sqrt{S^2/n}} \sim t_{n-1}\)
\end{enumerate}

\subsection{Approximatives Konfidenzintervall}
Wir können den zentralen Grenzwertsatz benutzen, um eine standardnormalverteilte ZV zu erhalten, und damit die Konfidenzintervalle zu bestimmen.

\section{Tests}
\begin{subbox}{Null- und Alternativhypothese}
	Die Nullhypothese \(H_0\) und die Alternativhypothese \(H_A\) sind zwei Teilmengen \(\Theta_0 \subseteq \Theta, \Theta_A \subseteq \Theta\) wobei \(\Theta_0 \cap \Theta_A \ne \varnothing\). Eine Hypothese heisst \textit{einfach}, falls die Teilmenge aus einem einzelnen Wort besteht; sonst \textit{zusammengesetzt}.
\end{subbox}

\begin{mainbox}{Definition Test}
	Ein Test ist ein Tupel \((T,K)\), wobei \(T\) eine \(ZV\) der Form \(T=t(X_1, \ldots, X_n)\) und \(K \subseteq \R\) eine deterministische Teilmenge von \(\R\) ist. Wir nennen \(T\) die \textit{Teststatistik} und \(K\) den \textit{Verwerfungsbereich} oder kritischen Bereich.
\end{mainbox}

Wir wollen nun anhand der Daten \((X_1(\omega), \ldots, X_n(\omega))\) entscheiden, ob die Nullhypothese akzeptiert oder verworfen wird. Zuerst berechnen wir die Teststatistik \(T(\omega) = t(X_1(\omega), \ldots, X_n(\omega))\) und gehen dann wie folgt vor:
\begin{itemize}
	\item Die Hypothese \(H_0\) wird \textit{verworfen}, falls \(T(\omega) \in K\).
	\item Die Hypothese \(H_0\) wird \textit{akzeptiert}, falls \(T(\omega) \notin K\).
\end{itemize}
\begin{subbox}{Fehler 1. und 2. Art}
	Ein Fehler 1. Art ist, wenn \(H_0\) fälschlicherweise verworfen wird, obwohl sie richtig ist.
	\[\P_\theta[T \in K], \quad \theta \in \Theta_0\]
	\noindent Ein Fehler 2. Art ist, wenn \(H_0\) fälschlicherweise akzeptiert wird, obwohl sie falsch ist.
	\[\P_\theta[T\notin K] = 1 - \P_\theta[T \in K], \quad \theta \in \Theta_A\]
\end{subbox}
\subsection{Signifikanzniveau und Macht}
Ein Test hat Signifikanzniveau \(a \in [0,1]\) falls
\[\forall \theta \in \Theta_0 \quad \P_\theta[T \in K] \le a\]
Es ist meist unser primäres Ziel, die Fehler 1. Art zu minimieren.

Das sekundäre Ziel ist, Fehler 2. Art zu vermeiden. Hierfür definieren wir die Macht eines Tests als Funktion:
\[\beta : \Theta_A \mapsto [0,1], \quad \theta \mapsto \P_\theta[T \in K]\]
Zu beachten ist, dass eine kleine Wahrscheinlichkeit für einen Fehler 2. Art einem \textit{grossen} \(\beta\) entspricht.

\subsection{Konstruktion von Tests}
Wir nehmen an, dass \(X_1, \ldots, X_n\) diskret oder gemeinsam stetig unter \(\P_{\theta_0}\) und \(\P_{\theta_A}\) sind, wobei \(\theta_0 \ne \theta_A\) einfach sind.

\noindent Der Likelihood-Quotient ist somit wohldefiniert:
\[R(x_1, \ldots, x_n) = \frac{L(x_1,\ldots, x_n;\theta_A)}{L(x_1, \ldots, x_n;\theta_0)}\]
(Falls \(L(x_1, \ldots, x_n; \theta_0) = 0\) setzen wir \(R(x_1, \ldots, x_n) = +\infty\).) Wenn \(R \gg 1\), so gilt \(H_A > H_0\) und analog \(R \ll 1 \implies H_A < H_0\).

\begin{subbox}{Likelihood-Quotient-Test}
	Der Likelihood-Quotient-Test (LQ-Test) mit Parameter \(c \ge 0\) ist definiert durch:
	\[T = R(x_1, \ldots, x_n) \quad \text{und} \quad K = (c, \infty]\]
\end{subbox}
Der LQ-Test ist optimal, da jeder andere Test mit kleinerem Signifikanzniveau auch eine kleinere Macht hat (Neyman-Pearson-Lemma).

\subsection{p-Wert}
Sei \(T = t(X_1, \ldots, X_n)\) eine Teststatistik und \((T,K_t)_{t\ge 0}\) eine Familie von Tests.

\begin{subbox}{Geordnete Teststatistik}
	Eine Familie von Tests heisst geordnet bzgl. \(T\) falls \(K_t \subset \R\) und \(s \le t \implies K_t \subset K_S\). Beispiele:
	\begin{itemize}
		\item \(K_t = (t, \infty)\) (rechtsseitiger Test)
		\item \(K_t = (-\infty, -t)\) (linksseitiger Test)
		\item \(K_t = (-\infty, -t) \cup (t, \infty)\) (beidseitiger Test)
	\end{itemize}
\end{subbox}

\begin{mainbox}{Definition p-Wert}
	Sei \(H_0: \theta = \theta_0\) eine einfache Nullhypothese. Sei \((T, K_t)_{t\ge 0}\) eine geordnete Familie von Tests. Der \(p\)-Wert ist definiert als ZV \(G(t)\), wobei
	\[G: \R_+ \mapsto [0,1], \quad G(t) = \P_{\theta_0}[T \in K_t]\]
\end{mainbox}
Der \(p\)-Wert hat folgende Eigenschaften:
\begin{enumerate}
	\item Sei \(T\) stetig und \(K_t = (t, \infty)\). Dann ist der \(p\)-Wert unter \(\P_{\theta_0}\) auf \([0,1]\) gleichverteilt.
	\item Für einen \(p\)-Wert \(\gamma\) gilt, dass alle Tests mit Signifikanzniveau \(\alpha > \gamma\) die Nullhypothese verwerfen.
\end{enumerate}

Insgesamt gilt also:
\[p\text{-Wert ist klein } \implies H_0 \text{wird wahrscheinlich verworfen} \]

\section{Quellen}
Dieses Cheatsheet wurde stark vom \href{https://n.ethz.ch/~dcamenisch/summaries}{Cheatsheet von Danny Camenisch} inspiriert. Ausserdem stammen Teile der Tabellen aus dem Buch ``Formeln, Tabellen und Konzepte''. Die Definitionen sind grösstenteils vom offiziellen Vorlesungsskript (V. Tassion) übernommen.


\clearpage
\section{Tabellen}

\subsection{Grenzwerte}
\begin{center}
	\begin{tabularx}{\linewidth}{XX}
		\toprule
		$\limxi \frac{e^x}{x^m} = \infty$                          & $\limxn xe^x = 0$                        \\
		$\limxi (1+x)^{\frac{1}{x}} = 1$                           & $\limxo (1+x)^{\frac{1}{x}} = e$         \\
		$\limxi (1+\frac{1}{x})^b = 1$                             & $\limxi n^{\frac{1}{n}} = 1$             \\
		$\limxo \frac{e^x-1}{x} = 1$ & $\limxi (1-\frac{1}{x})^x = \frac{1}{e}$ \\
		$\lim_{x\to\pm\infty} (1 + \frac{k}{x})^{mx} = e^{km}$     & $\limxi (\frac{x}{x+k})^x = e^{-k}$      \\
		$\limxo \frac{\log 1 - x}{x} = -1$            & $\limxo x \log x = 0$              \\
		$\limxo \frac{e^{ax}-1}{x} = a$               & $\limxo \frac{\ln(x+1)}{x} = 1$    \\
		$\lim_{x\to 1} \frac{\ln(x)}{x-1} = 1$        & $\limxi \frac{\log(x)}{x^a} = 0$   \\
		\bottomrule
	\end{tabularx}
\end{center}

\begin{mainbox}{Partielle Integration}
 \vspace{-12pt}
 $$\int f'(x) g(x) \mathop{dx} = f(x)g(x) - \int f(x) g'(x) \mathop{dx}$$
\end{mainbox}
\begin{itemize}
 \item Meist gilt: Polynome ableiten ($g(x)$), wo das Integral periodisch ist ($\sin, \cos, e^x$,...) integrieren ($f'(x)$)
 \item Teils: mit $1$ multiplizieren, um partielle Integration anwenden zu können (z.B. im Fall von $\int \log(x) \mathop{dx}$)
\end{itemize}
\begin{mainbox}{Substitution}
		Um $\int_a^b f(g(x)) \mathop{dx}$ zu berechnen: Ersetze $g(x)$ durch $u$ und integriere $\int_{g(a)}^{g(b)} f(u) \frac{\text{d}u}{g'(x)}$.
\end{mainbox}
\begin{itemize}
 \item $g'(x)$ muss sich herauskürzen, sonst nutzlos.
 \item Grenzen substituieren nicht vergessen.
 \item Alternativ: unbestimmtes Integral berechnet werden und dann $u$ wieder durch $x$ substituieren.
\end{itemize}

\subsection{Ableitungen}
\begin{center}
	% the c>{\centering\arraybackslash}X is a workaround to have a column fill up all space and still be centered
	\begin{tabularx}{\linewidth}{c>{\centering\arraybackslash}Xc}
		\toprule
		$\mathbf{F(x)}$                        & $\mathbf{f(x)}$          & $\mathbf{f'(x)}$         \\
		\midrule
		$\frac{x^{-a+1}}{-a+1}$                & $\frac{1}{x^a}$          & $\frac{a}{x^{a+1}}$      \\
		$\frac{x^{a+1}}{a+1}$                  & $x^a \ (a \ne 1)$        & $a \cdot x^{a-1}$        \\
		$\frac{1}{k \ln(a)}a^{kx}$             & $a^{kx}$                 & $ka^{kx} \ln(a)$         \\
		$\ln |x|$                              & $\frac{1}{x}$            & $-\frac{1}{x^2}$         \\
		$\frac{2}{3}x^{3/2}$                   & $\sqrt{x}$               & $\frac{1}{2\sqrt{x}}$    \\
		$-\cos(x)$                             & $\sin(x)$                & $\cos(x)$                \\
		$\sin(x)$                              & $\cos(x)$                & $-\sin(x)$               \\
		$\frac{1}{2}(x-\frac{1}{2}\sin(2x))$   & $\sin^2(x)$              & $2 \sin(x)\cos(x)$       \\
		$\frac{1}{2}(x + \frac{1}{2}\sin(2x))$ & $\cos^2(x)$              & $-2\sin(x)\cos(x)$       \\
		\multirow{2}*{$-\ln|\cos(x)|$}         & \multirow{2}*{$\tan(x)$} & $\frac{1}{\cos^2(x)}$    \\
																					 &                          & $1 + \tan^2(x)$          \\
		$\cosh(x)$                             & $\sinh(x)$               & $\cosh(x)$               \\
		$\log(\cosh(x))$                       & $\tanh(x)$               & $\frac{1}{\cosh^2(x)}$   \\
		$\ln | \sin(x)|$                       & $\cot(x)$                & $-\frac{1}{\sin^2(x)}$   \\
		$\frac{1}{c} \cdot e^{cx}$             & $e^{cx}$                 & $c \cdot e^{cx}$         \\
		$x(\ln |x| - 1)$                       & $\ln |x|$                & $\frac{1}{x}$            \\
		$\frac{1}{2}(\ln(x))^2$                & $\frac{\ln(x)}{x}$       & $\frac{1 - \ln(x)}{x^2}$ \\
		$\frac{x}{\ln(a)} (\ln|x| -1)$         & $\log_a |x|$             & $\frac{1}{\ln(a)x}$      \\
		\bottomrule
	\end{tabularx}
\end{center}
\subsection{Weitere Ableitungen}
\begin{center}
	\begin{tabularx}{\linewidth}{>{\centering\arraybackslash}X>{\centering\arraybackslash}X}
		\toprule
		$\mathbf{F(x)}$ & $\mathbf{f(x)}$             \\
		\midrule
		$\arcsin(x)$    & $\frac{1}{\sqrt{1 - x^2}}$  \\
		$\arccos(x)$    & $\frac{-1}{\sqrt{1 - x^2}}$ \\
		$\arctan(x)$    & $\frac{1}{1 + x^2}$         \\
		$x^x \ (x > 0)$ & $x^x \cdot (1 + \ln x)$     \\
		\bottomrule
	\end{tabularx}
\end{center}
\subsection{Integrale}
\begin{center}
	\begin{tabularx}{\linewidth}{>{\centering\arraybackslash}X>{\centering\arraybackslash}X}
		\toprule
		$\mathbf{f(x)}$                      & $\mathbf{F(x)}$                                                  \\
		\midrule
		$\int f'(x) f(x) \mathop{dx}$                & $\frac{1}{2}(f(x))^2$                                            \\
		$\int \frac{f'(x)}{f(x)} \mathop{dx}$        & $\ln|f(x)|$                                                      \\
		$\int_{-\infty}^\infty e^{-x^2} \mathop{dx}$ & $\sqrt{\pi}$                                                     \\
		$\int (ax+b)^n \mathop{dx}$                  & $\frac{1}{a(n+1)}(ax+b)^{n+1}$                                   \\
		$\int x(ax+b)^n \mathop{dx}$                 & $\frac{(ax+b)^{n+2}}{(n+2)a^2} - \frac{b(ax+b)^{n+1}}{(n+1)a^2}$ \\
		$\int (ax^p+b)^n x^{p-1} \mathop{dx}$        & $\frac{(ax^p+b)^{n+1}}{ap(n+1)}$                                 \\
		$\int (ax^p + b)^{-1} x^{p-1} \mathop{dx}$   & $\frac{1}{ap} \ln |ax^p + b|$                                    \\
		$\int \frac{ax+b}{cx+d} \mathop{dx}$         & $\frac{ax}{c} - \frac{ad-bc}{c^2} \ln |cx +d|$                   \\
		$\int \frac{1}{x^2+a^2} \mathop{dx}$         & $\frac{1}{a} \arctan \frac{x}{a}$                                \\
		$\int \frac{1}{x^2 - a^2} \mathop{dx}$       & $\frac{1}{2a} \ln\left| \frac{x-a}{x+a} \right|$                 \\
		$\int \sqrt{a^2+x^2} \mathop{dx} $           & $\frac{x}{2}f(x) + \frac{a^2}{2}\ln(x+f(x))$                     \\
		\bottomrule
	\end{tabularx}
\end{center}
\clearpage

\renewcommand*{\arraystretch}{2.5}
\subsection{Diskrete Verteilungen}
\begin{center}
	\begin{tabularx}{\textwidth}{llXXXX}
		\toprule
		Verteilung & Parameter & \( \E[X] \) & \( \Var(X) \) & \( p_X(t) \) & \( F_X(t) \) \\
		\midrule
		Gleichverteilung & \makecell[l]{\( n \): Anzahl Ereignisse \\ \( x_i \): Ereignisse} & \( \frac{1}{n} \sum_{i=1}^{n} x_i \) & \( \frac{1}{n} \sum_{i=1}^{n} x_i^2 - \frac{1}{n^2} \left(\sum_{i=1}^{n} x_i \right)^2 \) & \( \frac{1}{n} \) & \( \frac{|\{k:x_k \leq t\}}{n} \) \\

		Bernoulli & \( p: \) ErfolgsWK & \( p \) & \( p \cdot (1-p) \) & \( p^t(1-p)^{1-t} \) & \( 1-p \) für \( 0 \leq t < 1 \) \\

		Binomial & \makecell[l] {\( n \): Anzahl Versuche \\ \( p: \) ErfolgsWK } & \( np \) & \( np(1-p) \) & \( \binom{n}{t}p^t(1-p)^{n-t} \) & \( \sum_{k=0}^{t} \binom{n}{k} p^k(1-p)^{n-k} \)  \\

		Geometrisch & \makecell[l] { \( p \): ErfolgsWK \\ \( t: \) Anzahl Versuche} & \( \frac{1}{p} \) & \( \frac{1-p}{p^2} \) & \( p(1-p)^{t-1} \) & \( 1-(1-p)^t\) \\

		Poisson & \makecell[l]{ \( \lambda \): Erwartungswert \\ und Varianz} & \( \lambda \) & \( \lambda \) & \( \frac{\lambda^t}{t!}e^{-\lambda} \) & \( e^{-\lambda} \sum_{k=0}^{t} \frac{\lambda^{k}}{k!} \) \\

		\bottomrule

	\end{tabularx}
\end{center}

\bigskip
\subsection{Stetige Verteilungen}
\begin{center}
	\begin{tabularx}{\textwidth}{llXXXX}
		\toprule
Verteilung & Parameter & \( \E[X] \) & \( \Var(X) \) & \( f_X(t) \) & \( F_X(t) \) \\

\midrule
Gleichverteilung & \( [a,b] \): Intervall & \( \frac{a+b}{2} \) & \( \frac{1}{12}(b-a)^2 \) & \( \frac{1}{b-a} \) & \( \frac{t-a}{b-a} \) \\

Exponentialverteilung & \( \lambda: \frac{1}{\E[X]} \) & \( \frac{1}{\lambda} \) & \( \frac{1}{\lambda^2} \) & \( \begin{cases} \lambda e^{-\lambda t} & t \geq 0 \\ 0 & t < 0 \end{cases} \) & \( \begin{cases} 1-e^{-\lambda t} & t>0 \\ 0 & t \leq 0\end{cases}\) \\

Normalverteilung & \makecell[l]{\( \sigma^2 \): Varianz \\ \( \mu: \E[X] \)} & \( \mu \) & \( \sigma ^2 \) & \( \frac{1}{\sqrt{2\pi \sigma^2} }e^{-{\frac{(t-\mu)^2}{2\sigma^2} }} -\infty < t <\infty \) & \( \frac{1}{\sigma {\sqrt{2\pi}}} \int_{-\infty}^t e^{-\frac{1}{2}\left( \frac{y-\mu}{\sigma} \right) ^2} \mathrm{d} y \) \\ 

\( \chi ^2 \)-Verteilung & \( n \): Freiheitsgrad & \( n \) & \( 2n \) & \( \frac{1}{2^{\frac{n}{2}}\Gamma (\frac{n}{2})} t^{\frac{n}{2}-1} e^{-\frac{t}{2}} \quad t>0\) & Gamma\(\left( \frac{n}{2}, \frac{t}{2}\right)  \) \\

t-Verteilung & \( n \): Freiheitsgrad & \( \begin{cases} 0 & n>1 \\ \text{undef.} & \text{sonst} \end{cases} \) & \( \begin{cases} \frac{n}{n-2} & n> 2 \\ \infty & 1<n \leq 2 \\ \text{undef.} & \text{sonst} \end{cases} \) & \( \frac{\Gamma \left( \frac{n+1}{2} \right) }{\sqrt{n\pi } \cdot \Gamma (\frac{n}{2}}) \left( 1+ \frac{t^2}{n} \right) ^{- \frac{n+1}{2}} \) & I'd rather not \\


\bottomrule
	\end{tabularx}
\end{center}

\end{document}
